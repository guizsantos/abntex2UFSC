\documentclass[openright]{memoir}
\usepackage[utf8]{inputenc}
\usepackage[T1]{fontenc}
\usepackage[english,brazil]{babel}
\usepackage{minted,graphicx,multicol,xcolor,indentfirst,enumitem}
% Minted settings
\definecolor{codingbg}{rgb}{0.95,0.95,0.95}
\setminted{escapeinside=!!}
\newmintinline[code]{latex}{}
\newmint[displaycode]{latex}{}
\newminted[coding]{latex}{bgcolor=codingbg}
\makeatletter% minted fix for increasing separation between code blocks and text
\patchcmd{\minted@colorbg}{\noindent}{\medskip\noindent}{}{}
\apptocmd{\endminted@colorbg}{\par\medskip}{}{}
\makeatother

% Enumitem settings
\newlist{keys}{description}{3}
\setlist[keys]{format=\bfseries\ttfamily}
\newcommand{\key}[2][nenhum]{\item[#2] \hfill (padrão: \texttt{#1})\\}

% Other settings
\newcommand*{\abntex}{abn\TeX2}
\newcommand*{\abntexUFSC}{abn\TeX2UFSC}
\newcommand{\emingles}[1]{\foreignlanguage{english}{\textit{#1}}}

% Title
\title{\sffamily O pacote \abntexUFSC}
\date{\today}
\author{Guilherme Zanotelli dos Santos}
\usepackage[hidelinks]{hyperref}
\hypersetup{%
	pdfcreator={LaTeX}
	,pdfauthor={Guilherme Z. Santos}
	,pdftitle={O pacote abnTeX2UFSC}
}
\begin{document}
\pagestyle{simple}
\frontmatter
\maketitle
\begin{abstract}
Fazer um documento acadêmico de alta qualidade e em conformidade com as normas ABNT e da Biblioteca Universitária (BU) da Universidade Federal de Santa Catarina (UFSC) pode ser um fardo pesado demais para uns, que muitas vezes optam por abrir mão de um desses quesitos para focar/cumprir o outro. Como forma de contribuição para a comunidade acadêmica da UFSC e extender o trabalho do Dr. Lauro Césa5r, criador da classe \abntex, foi criado o pacote \abntexUFSC\ que formata a classe \abntex\ segundo as normas da BU, de maneira simples e flexível, visando reduzir o tempo desprendido em formatação permitindo o autor focar apenas no conteúdo sem abrir mão da qualidade tipográfica do documento. Este manual descreve todas as funções do pacote.
\end{abstract}
\tableofcontents*
\mainmatter
\chapter{Introdução}
De forma a facilitar a formatação da classe \abntex\ criou-se um único comando que através do sistema \texttt{\meta{chave}=\meta{valor}} é capaz de efetuar diversas tarefas, como ajustar o tamanho da fonte dos títulos de capítulo escrevendo \displaycode|chapter font size=\huge| em oposição ao que seria normalmente: \displaycode|\renewcommand{\ABNTEXchapterfontsize}{\huge}|

Além disso, é fez-se comandos para criar a capa dura, a capa normal (não colorida), e uma capa para fazer um PDF versão e-book. Comandos para criar a folha de aprovação, folha de rosto e etc. Toda essa configuração do \emingles{template} é feita no preambulo, dentro do comando \displaycode|\setUFSCtemplate{chave = valor, chave = valor}|

Entretanto, somente ao carregar o pacote diversas configurações já são automaticamente executadas, deixando o documento no estilo da BU, mas como algumas configurações são feitas pela classe devem ser passadas por ela.

\section{O \texorpdfstring{\abntexUFSC}{abnTeX2UFSC} para os apressados}
O pacote \abntexUFSC\ \textbf{deve ser} (obrigatoriamente) o \textbf{primeiro pacote} a ser carregado, Somente carrega-lo já garante o cumprimento com grande parte dos requisitos, restando apenas selecionar algumas opções da classe como \begin{enumerate}
\item o tamanho do papel
\item o tamanho da fonte
\item línguas do documento (como se fossem opções do babel)
\end{enumerate}

Atualmente a tendência é que todos os documentos passem a ser em formato A5 (\code|a5paper|) e fonte tamanho 10.5pt (\code|10.5pt|)\footnote{Esse tamanho não é usual do \LaTeX, ele foi criado pelos autores do pacote, portanto passar um valor arbitrário não funcionará, veja \autoref{fonte-qualquer} para saber como selecionar fontes de tamanhos qualquer}. Sendo assim, um típico preâmbulo para teses seria:

\begin{coding}
\documentclass[a5paper,10.5pt,english,brazil]{abntex2}
\usepackage{abntex2UFSC}

\setUFSCtemplate{%
	author   = {Nome Do Autor},
	title    = {Título do trabalho},
	subtitle = {Subtítulo},
	date     = {\thebook\year},
	local    = {Florianópolis}
}

\begin{document}
	\imprimircapa
	...
\end{coding}

O \abntexUFSC\ carrega os pacotes relevantes para a codificação do documento de acordo com o compilador usado, ou seja, caso compile-se usando \texttt{pdflatex} os pacotes \texttt{inputenc} e \texttt{fontenc} são carregados, e caso o \texttt{xelatex} seja usado, o \texttt{fontspec} é carregado, não sendo necessário carregar nenhum desses pacotes.

Outros pacotes carregados pelo \abntexUFSC:
\begin{multicols}{3}
\begin{itemize}
\item pgfkeys \item pdfpages    \item varwidth \item fix-cm       \item tikz \item geometry
\item xcolor  \item indentfirst \item booktabs \item afterpackage \item calc
\end{itemize}
\end{multicols}

\chapter{Configurando o \textit{template}}
Dentro do comando \code|\setUFSCtemplate| existem várias chaves que podem ser configuradas, neste capítulo trataremos de cada uma delas.

\section{Formatando títulos}
As chaves que fazem a formatação dos títulos são bastante intuitivas.

\begin{keys}
\key[\code|\normalfont\bfseries|]{part font} Configura o estilo da fonte das partes
\key[\code|\normalfont\bfseries|]{chapter font} Configura o estilo da fonte dos capítulos
\key[\code|\normalfont|]{section font} Configura o estilo da fonte das seções
\key[\code|\normalfont\bfseries|]{subsection font} Configura o estilo da fonte das subseções
\key[\code|\normalfont|]{subsubsection font} Configura o estilo da fonte das subsubseções
\key[\code|\normalfont\itshape|]{subsubsubsection font} Configura o estilo da fonte das subsubsubseções
\key[\code|\normalfont\bfseries|]{part font size} Configura o tamanho da fonte das partes
\key[\code|\normalsize|]{chapter font size} Configura o tamanho da fonte dos capítulos
\key[\code|\normalsize|]{section font size} Configura o tamanho da fonte das seções
\key[\code|\normalsize|]{subsection font size} Configura o tamanho da fonte das subseções
\key[\code|\normalsize|]{subsubsection font size} Configura o tamanho da fonte das subsubseções
\key[\code|\normalsize|]{subsubsubsection font size} Configura o tamanho da fonte das subsubsubseções
\key[-]{uppercased chapters} Coloca os capítulos em caixa alta, como trata-se de um estilo ele não recebe argumento
\key[-]{uppercased sections} Coloca as seções em caixa alta
\key[-]{uppercased subsections} Coloca as subseções em caixa alta
\key[-]{uppercased subsubsections} Coloca as subsubseções em caixa alta
\key[-]{uppercased subsubsubsection} Coloca as subsubsubseções em caixa alta
\key[\code|\normalsize|]{titles font size} Coloca todos os títulos (desde \texttt{part} à \texttt{subsubsubsection}) no mesmo tamanho de fonte
\end{keys}

Os estilos possíveis são quaisquer estilos  típicos de fonte, como por exemplo:
\begin{itemize}
\item \code|\rmfamily| coloca o texto em fonte Romana (com serifas);
\item \code|\normalfont| coloca o texto em fonte ``normal'', usualmente é equivalente à \code|\rmfamily|;
\item \code|\sffamily| coloca o texto em fonte \textsf{Sem serifa} (\textit{sans serif}), fonte tipo Arial;
\item \code|\scshape| coloca o texto em fonte \textsc{Versalete};
\item \code|\itshape| coloca o texto em fonte \textit{Italicizada}.
\item \code|\bfseries| coloca o texto em fonte \textbf{Negrito}.
\end{itemize}

Quanto aos tamanhos de fonte, apesar de poderem ser configurados manualmente, é aconselhado que utilize-se dos padrões do \LaTeX:

\begin{itemize}
\item \code|\tiny| {\tiny muito pequeno};
\item \code|\scriptsize| {\scriptsize tamanho de subscrito};
\item \code|\footnotesize| {\footnotesize tamanho de nota de rodapé};
\item \code|\small| {\small pequeno};
\item \code|\normalsize| {\normalsize normal} (tamanho do documento);
\item \code|\large| {\large um pouco maior};
\item \code|\Large| {\Large maior};
\item \code|\LARGE| {\LARGE bem maior};
\item \code|\huge| {\huge enorme};
\item \code|\Huge| {\Huge gigante};
\end{itemize}

\subsubsection{Fontes de qualquer tamanho}
\label{fonte-qualquer}

Para configurar uma fonte em um tamanho arbitrário, primeiramente deve-se saber se essa fonte suporta o tamanho requisitado, no caso de fontes escaláveis (True Type, por exemplo) qualquer tamanho é suportado, mas algumas fontes mais antigas, como a fonte nativa do \LaTeX\ (\emph{Computer Modern Roman}) não são escaláveis e acabam ficando ``feias'' quando colocadas em tamanhos que não foram configuradas. Felizmente o pacote \texttt{fix-cm} corrige os problemas com a fonte nativa do \LaTeX permitindo que escolhamos qualquer tamanho de fonte.

Para selecionar um tamanho qualquer usa-se o comando \displaycode|\fontesize!\marg{size}\marg{skip}!\selectfont| Onde \meta{size} é o tamanho da fonte (em pt) e \meta{skip} é o \emingles{baseline skip} (a distância entre linhas, usualmente adota-se um valor de 1,2 vezes o tamanho da fonte). O comando \code|\selectfont| serve para selecionar o tamanho de fonte escolhido. Aplicando-se à maneira como o \abntexUFSC\ funciona: \displaycode|chapter font size={\fontsize{16pt}{19.2pt}\selectfont}|
\section{Informações de capa e afins}
Vários dos elementos pré-textuais podem ser automaticamente gerados desde que tenham sido fornecidas as informações adequadas. Por exemplo, a capa necessita das seguintes informações: nome do autor, título do trabalho, data e local. As chaves que guardam essas informações são:

\begin{keys}
\key{author} Guarda o autor do trabalho (para múltiplos autores usar \code|\and| ou \code|\\|).
\key{title} Guarda o título do trabalho.
\key{subtitle} Guarda o subtítulo do trabalho.
\key{date} Guarda a data (pode ser só o ano como \code|\the\year|)
\key{local} Guarda o local onde o trabalho foi realizado.
\key[Universidade Federal de Santa Catarina]{institution} Guarda a instituição em que o trabalho foi realizado.
\key{advisor} Guarda o orientador.
\key{co-advisor} Guarda o co-orientador.
\key{coordinator} Guarda o coordenador do programa (no caso de TCC é o professor da disciplina)
\key[-]{nature of work} Configura a natureza do trabalho, e gera automaticamente o texto da folha de rosto e da folha de aprovação com as informações que são dadas. Aqui são válidas as seguintes chaves:\footnote{Essa chave é um tanto atípica com relação às outras, dentro dela são válidas outras chaves}
	\begin{keys}
	\key{type} Pode assumir os valores: \texttt{tcc}, \texttt{monography}, \texttt{masters dissertation}, \texttt{doctoral thesis} e \texttt{post-doc report}. Configura automaticamente o texto do tipo de trabalho e a cor da capa segundo o requisitado pela BU.
	\key{course} Guarda o nome do curso/programa responsável pela formação acadêmica.
	\key{given title} Guarda o nome do título recebido pela conclusão do trabalho desenvolvido.
	\end{keys}
\key{nature of work text} Guarda o texto colocado na folha de rosto, caso o texto padrão gerado pelas informações dadas em \texttt{nature of work} não satisfaça, pode-se sobrescreve-lo.
\key[none]{approval sheet text} Guarda o texto que deve ser usado na folha de aprovação, fornecido pelo programa, por padrão caso o texto seja \texttt{none} usa-se um texto modelo.
\key{bench N} Guarda o nome do membro da banca, \texttt{N} pode ser um número inteiro de \texttt{1} a \texttt{4}. Para mais membros de banca veja o \autoref{customização}.
\key[6cm]{sign line width} Configura a largura da linha de assinatura.
\key[1pt]{sign line thickness} Configura a espessura da linha de assinatura.
\key[0.75cm]{sign skip} Configura a distância entre a linha de assinatura e o nome.
\key{hard cover} Configura informações que serão utilizadas na confecção. Aqui são válidas as seguintes chaves:
	\begin{keys}
	\key[\code|\imprimirtitulo|]{cover text} 
	\key[\code|\imprimirpreambulo\\2\baselineskip\imprimirorientadorRotulo:~\imprimirorientador|]{counter cover text} 
	\key{cover flap text} 
	\key{counter cover flap text}
	\key{lower bar text}
	\key[0.075]{X}
	\end{keys}
\end{keys}


\section{Fugindo da norma}

\chapter{Comandos do pacote}
\section{Gerando os elementos pré-textuais}


\chapter{Alterando os padrões}
\label{customização}
\section{Capa}
\section{Folha de rosto}


\chapter{Dicas e boas práticas}
\section{Organização}
\section{Compiladores}
\subsubsection{-shell-escape}
\subsection{Bibliografia}
\subsection{Lista de siglas -- \texttt{glossaries}}
\section{Números e unidades}
\end{document}
