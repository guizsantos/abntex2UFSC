\documentclass[12pt,a4paper,oneside,english,brazil]{abntex2}
\usepackage{abntex2UFSC}

\usepackage{%
	calligra%       Fonte caligráfica da epigrafe
	,indentfirst%   Coloca os primeiros parágrafos com indentação
	,booktabs%      Implementa comandos para deixar as tabelas mais bonitas (\toprule \midrule \cmidrule \bottomrule ...)
	,graphicx%      Para inserir imagens
	,csquotes, lipsum}

\usepackage[acronym,nomain,nogroupskip,nonumberlist,noredefwarn,section=subsection]{glossaries}
\glsdisablehyper
\glssanitizesorttrue
\renewcommand{\glsprestandardsort}[3]{%
	\glsfieldfetch{#3}{desc}{#1}%
	\glsdosanitizesort
}

\setUFSCtemplate{%
	title=Título do trabalho de conclusão,
	subtitle=subtítutlo do trabalho,
	author=Autor do Trabalho,
	date=\the\year,
	local=Florianópolis,
	advisor={Prof. Orientador, Ph.D.},
	%co-advisor=Eng. Ian Jackes,
	coordinator={Prof. Coordenador do curso, Dr.},
	nature of work={
		type=masters dissertation,
		course={Curso de Graduação},
		given title=Mestre O-que-for
	},
	catalogue record={./ficha.pdf},
	bench1={Prof. Antônio Barcelos, Ph.D.\par Universidade Federal de Santa Catarina\par },
	bench2={Profª Carla Duarte, Dr.\par Universidade Federal do Amapá\par },
	bench3={Msc. Eduardo Fagundes\par Universidade Federal do Paraná\par },
	dedication={Dedico este trabalho àqueles que gostam do \LaTeX},
	epigraph={Do not go gentle into that good night}{Dylan Thomas, 1951. Em Interestellar 2014},
	epigraph font=\calligra,
	sign line width=8cm,
	sign line skip=0.75cm,
	easy upright indices={with ()},% Faz os sub e super escritos do modo matemático ficarem com "\text{}" (fonte normal -- não matemática) quando envoltos por "()" (Ex: X_(sup)  seria o mesmo que escrever X_{\text{sup})
%	use glossaries,% Essa chave configura um formato para as listas de siglas e símbolos, permite usar a chave 'unit' quando definindo uma entrada do glossaries e já passa o comando \makeglossaries. Também cria os comandos \imprimirlistadesiglas \imprimirlistadesimbolosromanos \imprimirlistadesimbolosgregos e \imprimirlistadesiglasesimbolos (esse último é equivalente a mandar imprimir os três anteriores)
	%Existe também a chave 'use nomencl' que faz configurações parecidas para quem quiser usar o nomencl ao invés do glossaries
}


\newglossaryentry{perimeter}{type=romansymbol,name={\ensuremath{P}},description={Perímetro},unit={\si{\meter}}}
\newglossaryentry{diameter}{type=romansymbol,name={\ensuremath{D}},description={Diâmetro},unit={\si{\meter}}}
\newglossaryentry{pi}{type=greeksymbol,name={\ensuremath{\pi}},description={Razão \gls{perimeter}/\gls{diameter}},unit={-}}
\newacronym[plural={EDOs},firstplural={equações diferenciais ordinárias (EDOs)},description={Equação Diferencial Ordinária}]{EDO}{EDO}{equação diferencial ordinária}
\newglossaryentry{ambient}{type=indices,name={amb},description={Ambiente}}
\newglossaryentry{prandtl}{type=numbers,name={Pr},description={Número de Prandtl},definition={$\dfrac{\nu c_p}{\rho}$}}

%Ajustando os floats:
\setfloatadjustment{table}{\centering\small} %Coloca todos os floats tipo table centralizados e com fonte pequena!
\setfloatadjustment{figure}{\centering} %Coloca todos os floats tipo figure centralizados
\setfloatlocations{table}{htb} %Configura os floats com os posicionadores htb (o padrão é tb, com o h agora os floats podem estar no meio do texto)
\setfloatlocations{figure}{htb}

% Configuração da bibliografia -- Deve ser feito por último!!!
\usepackage[backend=biber,
			style=abnt,
			giveninits,
			uniquename=false,
			uniquelist=false,
			doi=false,
			isbn=false,
			maxcitenames=2,
			maxbibnames=11,
			extrayear,
			noslsn,
			repeatfields,
			hyperref=false]{biblatex}
\addbibresource{referencias.bib}
\setlength\bibitemsep{\baselineskip}

\begin{document}
	\imprimircapa% Versão de capa normal, para capa dura usar \imprimircapa*
	\imprimirfolhaderosto
	\imprimirfichacatalografica
	%%Colocar a errata aqui (se tiver, é opcional)
	\imprimirfolhadeaprovacao
	\imprimirdedicatoria
	\begin{agradecimentos}
		Agradecimentos
	\end{agradecimentos}
	\imprimirepigrafe
	\begin{resumo}
		Resumo do trabalho de conclusão
	\end{resumo}
	\begin{resumo}[Abstract]
		Abstract of the present work
	\end{resumo}

	\pdfbookmark[0]{Nomenclatura/Lista de símbolos}{nom}
	\chapter*{Nomenclatura/Lista de símbolos}
	Lista de símbolos. Feita à mão ou via pacotes como o glossaries ou nomenclature. Nesse caso, feito com o glosssaries:
	
	\printglossaries
	
%	\pdfbookmark[1]{Siglas}{sig}
%	\printglossary[type=\acronymtype,style=siglas]
%	\pdfbookmark[1]{Símbolos romanos}{rom}
%	\printglossary[type=romansymbol,style=simbolos]
%	\pdfbookmark[1]{Símbolos gregos}{gre}
%	\printglossary[type=greeksymbol,style=simbolos]
%	\pdfbookmark[1]{Sub e superíndices}{ind}
%	\printglossary[type=indices,style=indices]
%	\pdfbookmark[1]{Números característicos}{num}
%	\printglossary[type=numbers,style=numbers]
	\cleardoublepage

	
	\pdfbookmark[0]{\contentsname}{toc}
	\tableofcontents*
	\cleardoublepage
	
	\textual
	
	\chapter{Introdução}
	Introdução do trabalho \cite{fox2010}. Como feito por \textcite{article}...
	
	O número \gls{pi}, é encontrado em \glspl{EDO}, e representa a razão entre o diâmetro, \gls{diameter}, e o perímetro ambiente ($\gls{perimeter}_{\gls{ambient}}$). Uma \gls{EDO} comum de ser encontrada dá origem ao número de Prandtl, \gls{prandtl}.
	
	\lipsum[1-8]
	\nocite{*}
	
	\chapter{Desenvolvimento}
	\lipsum[10]
	\section{Sub}
	\lipsum[11]
	\subsection{Subsub}
	\lipsum[12]
	
	\chapter{Conclusão}
	\lipsum[13-16]
	
	\postextual
	\printbibliography[title=\texorpdfstring{\MakeTextUppercase{\bibname}}{\bibname}]
	\apendices
	\chapter{Apêndices}
	Apêndices: aquilo de autoria própria e que é relevante para o trabalho mas não contribuí significativamente para estar no corpo principal do trabalho (deduções matemáticas, detalhamento de um assunto específico, etc)
	
	\anexos
	\chapter{Anexos}
	Anexos: aquilo de \textbf{não é de autoria própria} e que é relevante para o trabalho mas não contribuí significativamente para estar no corpo principal do trabalho (desenhos técnicos, especificações técnicas, etc)
\end{document}